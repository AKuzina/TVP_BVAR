Рассмотрим одномерные модели, которые были выбраны в качестве конкурентов для основной модели работы --- байесовской векторной авторегрессии с переменными коэффициентами и стохастической волатильностью. Этими одномерными моделями являются экспоненциальное сглаживание, ARIMA, тета-метод и их комбинации (гибрид из нескольких одномерных моделей). Подробнее процедура формирования гибридных моделей будет описана в разделе \ref{sec:method}, посвященному методологии работы.

\subsection{Экспоненциальное сглаживание} \label{sec:ets}
%%%%%%%%%% Простое экспоненциальное сглаживание
Семейство моделей экспоненциального сглаживания достаточно часто используются для прогнозирования временных рядов. Можно предположить, что они стали популярными, отчасти, благодаря своей простоте и гибкости. В основе всех моделей данного семейства лежит простое экспоненциальное сглаживание --- метод прогнозирования временных рядов, базирующийся на предположении, что каждое последующее наблюдение временного ряда может быть представлено как взвешенная сумма всех прошлых наблюдений. Причем, чем дальше наблюдение от последнего доступного, тем меньше его вес. Таким образом, последовательность весов представляет собой убывающую геометрическую прогрессию. Если записать описанный выше метод в виде математического выражения, то мы получим следующее уравнение для определения прогноза временного ряда:
\begin{multline} \label{ets_forecast}
\hat{y}_{T+1|T} = \alpha y_T + \alpha(1-\alpha)y_{T-1} + \alpha (1-\alpha)^2 y_{T-2} + ... = \\ 
 = \alpha y_{T} + (1-\alpha)\hat{y}_{T|T-1} 
\end{multline}

Так как уравнение \eqref{ets_forecast} позволяет получить прогноз только на 1 период вперед, для любого горизонта $h > 1$ прогноз будет равен одному и тому же значению, прогнозу для $h = 1$:
\begin{equation*}
\hat{y}_{T+h|T} = \hat{y}_{T+1|T}
\end{equation*}

Если записать уравнение \eqref{ets_forecast} для каждого доступного наблюдения, а потом подставить выражение для каждого сглаженного значения в предшествующее ему уравнение мы получим следующее выражение для прогноза на один период:
\begin{equation*} 
\hat{y}_{T+1|T} = \sum_{i = 0}^{T-1}\alpha(1-\alpha)^iy_{T-i} + (1-\alpha)^T \hat{y}_{1|0}
\end{equation*}

В результате, для того, чтобы получить экспоненциально сглаженный ряд, нам необходимо задать два параметра: начальное значение сглаженного ряда $\hat{y}_{1|0}$ (прогноз первого доступного наблюдения) и параметр $\alpha$.

% Начальное значение часто полагают равным первому доступному наблюдению $Y_1$. Другой подход, который можно использовать для оценивания как сглаживающего параметра $\alpha$, так и начального значения --- минимизация квадрата ошибок прогноза:
% \begin{equation*}
% RSS = \sum_{t = 1}^{T}(y_t - \hat{y}_{t|t-1})^2 \longrightarrow \underset{\alpha, \hat{y}_0}{min} 
% \end{equation*}
% 
% Решив задачу нелинейной оптимизации, мы получим оптимальные значения парметров и сможем построить прогноз временного ряда, используя уравнение \eqref{ets_forecast}.
%%%%%%%%%% Переход к модели пространства состояний
Если есть необходимость получить не только точечные прогнозы, но и их доверительные интервалы, нам необходимо определиться со статистической моделью, которая (как мы предполагаем) описывает наш временной ряд ${y_t}$. В случае с экспоненциальным сглаживанием можно построить модель, состоящую из двух основных компонентов: уравнения измерения (\EN{measurment equation}), описывающего наблюдаемые данные, и уравнений перехода (\EN{transition equations}), которые описывают ненаблюдаемые компоненты данных (например, тренд или сезонность). Модели, состоящие из таких элементов называют моделями пространства состояний (\EN{state space models}).

Все что нужно, чтобы перейти от детерминированных уравнений, описанных выше, к стохастической модели пространства состояний, это задать распределение ошибок $\varepsilon_t$. Предположив, что ошибки независимы и нормальны, то есть $\varepsilon_t = y_{t} - \hat{y}_{t|t-1} \sim i.i.d. N(0, \sigma^2)$, а также обозначив сглаженное значение (уровень ряда) $l_{t-1} = \hat{y}_{t|t-1}$, получим следующую модель:
\begin{align}
\begin{split} \label{ssm_simple_ets}
y_t &= l_{t-1} + \varepsilon_t\\
l_t &= l_{t-1} + \alpha \varepsilon_t
\end{split}
\end{align}

%%%%%%%%%% Мультипликативные ошибки
Если же вместо обычной ошибки прогноза принять $\varepsilon_t$ равным относительной ошибке, то мы получим модель с мультипликативными ошибками, которая имеет немного другой формат записи.
\begin{align*}
y_t = l_{t-1} &(1 + \varepsilon_t)\\
l_t = l_{t-1}&(1 + \alpha \varepsilon_t),\\
\quad  \text{где} \quad &\varepsilon_t = \frac{y_t - \hat{y}_{t|t-1}}{\hat{y}_{t|t-1}} \sim i.i.d N(0,\sigma^2)
\end{align*}

Заметим, однако, что значение точечных прогнозов не зависит от вида ошибок, то есть при использовании модели с аддитивными и мультипликативными ошибками будут получаться совершенно одинаковые точечные прогнозы. Отличие будет заметно только за счет разных доверительных интервалов. 

Стоит заметить, что простое экспоненциальное сглаживание может показывать очень плохие результаты, особенно когда оно имеет дело с данными, обладающими трендом или сезонностью. Именно по этой причине существуют другие, более продвинутые версии данной модели. Начнем с метода Хольта, который используется для прогнозирования данных с линейным трендом. В простейшем случае предполагается, что тренд входит в модель аддитивно, то есть появляется новый компонент в уравнении измерений, а также добавляется еще одно уравнение переходов, описывающее изменение тренда во времени:
\begin{align}
\begin{split} \label{ssm_holt}
y_t &= l_{t-1} + b_{t-1} + \varepsilon_t\\
l_t &= l_{t-1} + b_{t-1} + \alpha \varepsilon_t\\
b_t &= b_{t-1} + \beta \varepsilon_t
\end{split}
\end{align}

В данном случае прогноз уже не будет горизонтальным, так как к уравнению уровня добавится линейный тренд:
\begin{equation*}
y_{T+h|T} = l_T + hb_t
\end{equation*}

Таким образом, для получения прогноза нам нужно определить оптимальное значение параметров $\alpha, \beta$, а также начальные значения тренда и уровня --- $l_0, b_0$. Помимо такой спецификации, выделяют также модель с экспоненциальным трендом. В этом случае компонента $b_t$ входит в уравнение измерений мультипликативно. Более того, существуют модели с демпфированным трендом (\EN{damped trend}), в которых по мере увеличения горизонта прогноза снижается и вклад трендовой составляющей, что позволяет получать более правдоподобные долгосрочные прогнозы.

Помимо тренда, временные ряды часто обладают свойством сезонности, которую крайне важно учитывать при прогнозировании. Модель Хольта-Уинтерса позволяет получить экспоненциально сглаженный ряд с учетом наличия аддитивной или мультипликативной сезонности в данных. Добавление сезонности в модель, так же как и в случае с трендом, всегда означает появление нового компонента в уравнении измерений (входящего аддитивно или мультипликативно), а также нового уравнения переходов. Для примера рассмотрим, как изменится модель с линейным трендом \eqref{ssm_holt}, если в нее добавить мультипликативную сезонность.
\begin{align*}
\begin{split} 
y_t &= (l_{t-1} + b_{t-1})s_{t-m} + \varepsilon_t\\
l_t &= l_{t-1} + b_{t-1} + \alpha \varepsilon_t/s_{t-m}\\
b_t &= b_{t-1} + \beta \varepsilon_t/s_{t-m}\\
s_t &= s_{t-m} + \gamma \varepsilon_t / (l_{t-1} + b_{t-1})
\end{split}
\end{align*}

В данной работе, при прогнозировании методом экспоненциального сглаживания, используется функция \texttt{ets()} пакета \texttt{forecast}, написанного для языка программирования \texttt{R}. Для получения оценок параметров модели и начальных значений тренда и сезонности (если это необходимо) в пакете используется метод максимального правдоподобия, максимизирующий вероятность получения наблюдаемой выборки при заданных ограничениях. Ограничения задаются таким образом, чтобы сглаживающие параметры уровня, тренда и сезонности лежали в пределах $(0,1)$.

Заметим, что комбинируя различные виды тренда и сезонности можно получить 15 различных моделей, в каждой из которых могут быть мультипликативные или аддитивные ошибки. Классификация всех возможных комбинаций может быть представлена в виде таблицы.

\begin{center*}
\begin{table}[h]
\renewcommand{\arraystretch}{1.5}
\begin{threeparttable}
\centering
\caption{Классификация моделей экспоненциального сглаживания}
\label{ets_types}
\begin{tabularx}{\textwidth}{Xccc}
\toprule
\multicolumn{1}{c}{\multirow{2}{*}{Тренд}} & \multicolumn{3}{c}{Сезонность}  \\ \cmidrule(l){2-4} 
\multicolumn{1}{c}{}              &\footnotesize{Отсутствует} & \footnotesize{Аддитивная}&\footnotesize{Мультипликативная}\\ \midrule
\small{Отсутствует}        &  $NN$                     &  $NA$                    &  $NM$    \\
\small{Аддитивный}         &  $AN$                     &   $AA$                   &  $AM$    \\
\small{Аддитивный демпфированный}       & $A_dN$                    &   $A_dA$                 &  $A_dM$    \\
\small{Мультипликативный}&   $MN$                    &  $MA$                    &  $MM$    \\
\small{Мультипликативный демпфированный}&   $M_dN$                  &  $M_dA$                  &  $M_dM$    \\ \bottomrule
\end{tabularx}
\begin{tablenotes}
      \footnotesize 
      \item Источник: Hyndman, Koehler, Snyder, and Grose (2002), стр. 440
\end{tablenotes}
\end{threeparttable}

\end{table}
\end{center*}


Для выбора оптимальной спецификации используется информационный критерий Акаике, скорректированный на размер выборки, формула для расчета которого представлена в уравнении \eqref{AIC}. Оптимальной считается та модель, значение информационного критерия для которой минимально.
\begin{equation}\label{AIC}
AIC_c = -2\log(L) + 2k + \frac{2k(k+1)}{T-k-1},
\end{equation}
где $L$ --- значение функции правдоподобия, а k --- общее число коэффициентов в модели (включая начальные значения уровня, тренда и сезонности, если они есть).
\subsection{ARIMA} \label{sec:arima}
ARIMA представляет собой комбинацию авторегрессии и модели скользящего среднего. Полная спецификация модели \EN{$ARIMA(p, d, q)$}, где $p$ --- порядок авторегрессионной части, $q$ --- порядок части скользящего среднего и $d$ --- порядок разности ряда, выглядит следующим образом:
\begin{equation}\label{arima}
\Delta:в^d y_t = \alpha + \beta_1 \Delta^d y_{t-1} + \cdots + \beta_p \Delta^d y_{t-p} + \varepsilon_t + \gamma_1 \varepsilon_{t-1} + \cdots + \gamma_q \varepsilon_{t-q}
\end{equation}
Можно записать модель с помощью оператора лага $L$:\\
 \begin{equation*}
(1-L)^d (1 - \beta_1 L - \cdots - \beta_p L^p)y_t =\alpha + (1 + \gamma_1 L + \cdots + \gamma_q L^q)\varepsilon_t
\end{equation*}

Существует также разновидность моделей ARIMA, использующаяся для моделирования сезонных данных. Сезонная модель получается в результате добавления сезонных разностей в исходную модель \eqref{arima}. Такая модель обычно обозначается как $ARIMA(p,d,q) (P,D,Q)_m$, где $m$ --- частотность ряда (например, $m = 4$ для квартальных данных). Она также может быть записана при помощи оперетора лага. В результате, наиболее обощенная версия модели ARIMA имеет следующий вид:
\begin{align*}
(1-L)^d(1-L^m)^D (1 - \beta_1 L - \cdots - \beta_p L^p)(1-B_1L^m - \cdots - B_P L^{Pm} ) y_t =\\
= \alpha + (1 + \gamma_1 L + \cdots + \gamma_q L^q)(1 + \Gamma_1 L^m + \cdots + \Gamma_Q L^{Qm} ) \varepsilon_t
\end{align*}

Для определения параметра $d$ можно провести тест на единичный корень. В функции \texttt{auto.arima()} используется KPSS тест \EN{(Kwiatkowski, Phillips, Schmidt, Shin)}, с помощью которого определяется минимальное число разностей. Авторы пакета в своей статье \EN{Hyndman and Khandakar (2008)} утверждают, что  для автоматической оценки модели гораздо эффективнее использовать тесты на единичный корень, в которых нулевая гипотеза предполагает отсутствие единичного корня, так как тесты с нулевой гипотезой о существовании единичного корня ведут к смещенным результатам. KPSS тест предполагает, что рассматриваемый временной ряд имеет следующий вид:
\begin{align*}
y_t &= c_t + \beta t + u_t\\
c_t &= c_{t-1} + \varepsilon_t, \text{где}\quad \varepsilon_t \sim WN(0,\sigma^2)
\end{align*}
Такой ряд является стационарным, если $c_t$ постоянен во времени, то есть дисперсия $\sigma^2 = 0$ нулевая и альтернативная гипотезы формулируются следующим образом:
\begin{align*}
&H_0: \sigma^2 = 0\quad (\text{временной ряд стационарен})\\
&H_a: \sigma^2 \neq 0 \quad (\text{временной ряд не является стационарным})
\end{align*}
Тестовая статистика для проверки выдвинутой гипотезы считается следующим образом:
\begin{equation*}
KPSS = \frac{\sum_{i=1}^{T}S_t^2}{T^2 \lambda^2},
\end{equation*}
где $S_t = \sum_{j=1}^{t} \hat{u}_j$ --- сумма остатков в регрессии, а $\lambda^2$ --- состоятельная оценка дисперсии $u_t$ в форме Ньюи-Уэста. Сравнив полученное значение тестовой статистики с асимптотическим распределением, можно сделать вывод о правдоподобности нулевой гипотезы. Если гипотеза о стационарности отвергается, то берется первая разность исходного ряда и тест повторяется еще раз. Оптимальное значение параметра $d$ определяется как минимальное число разностей ряда, необходимое для того, чтобы нулевая гипотеза о его стационарности не отвергалась.

Для определения порядка сезонной разности в функции \EN{\texttt{auto.arima()}} используется OSCB тест. Данный тест, представленный в статье \EN{Osborn, Chui, Smith, Birchenhall (1988)}, заключается в проверке нулевой гипотезы о наличии сезонного единичного корня. Он был выбран авторами пакета, так как позволил получить модели с лучшими прогнозами, по сравнению с остальными тестами для определения параметра $D$. Допустимые значения сезонной разности ограничены множеством $\{0,1\}$, в результате чего тест на сезонный единичный корень проводится только один раз.

Определив параметры $d$ и $D$ описанными выше способами, функция \EN{\texttt{auto.arima()}} выбирает оптимальный порядок $p$,$q$, $P$, $Q$, минимизируя скорректированный на размер выборки информационный критерий Акаике, который аналогичен данному критерию из моделей экспоненциального сглаживания (см. уравнение \eqref{AIC}). Число коэффициентов в модели (включая дисперсию остатков) можно определить следующим образом:
\begin{align*}
k = p + q + c + 1,
\end{align*}
где $c$ --- дамми, отвечающая за наличие свободного коэффициента в модели.

Стоит, однако, заметить, что даже если выбирать из ограниченного числа параметров, количество моделей, для которых необходимо посчитать значение критерия Акаике, очень велико. Чтобы процесс автоматического выбора модели не требовал огромных вычислительных мощностей, авторы пакета предложили использование ступенчатого подхода, состоящего из следующих шагов:
\begin{description}
\item[Шаг 1] Оцениваются 4 модели:
\begin{itemize}
\item $\texttt{ARIMA}(2,d,2)(1,D,1)_m $;
\item $\texttt{ARIMA}(0,d,0)(0,D,0)_m$ ;
\item $\texttt{ARIMA}(1,d,0)(1,D,0)_m $;
\item $\texttt{ARIMA}(0,d,1)(0,D,1)_m $.
\end{itemize}
Отметим, что если $m = 1$, то рассматриваются модели без сезонной составляющей. Свободный коэффициент $\alpha$ включается в модель только в том случае, если $d + D \leq 1$. Из четырех указанных моделей выбирают <<текущую>> модель, которая характеризуется наименьшим значением $AIC_c$.

\item[Шаг 2] Рассматриваются 13 модификаций текущей модели, :
\begin{itemize}
\item один из параметров $p$, $q$, $P$, $Q$ отклоняется на 1 от текущего значения;
\item параметры $p$ и $q$ одновременно отклоняются на 1 от текущего значения;
\item параметры $P$ и $Q$ одновременно отклоняются на 1 от текущего значения;
\item в модель включается свободный коэффициент, если его не было в текущей модели, или оценивается модель без свободного коэффициента, если в текущей модели он был.
\end{itemize}
Если в результате данных модификаций была найдена модель с меньшим значением $AIC_c$, то она становится текущей и шаг 2 повторяется еще раз. Если у всех модифицированных моделей значение выше текущего, то процесс выбора завершается.
\end{description}

Наконец, при заданных $p, d, q$, методом максимального правдоподобия определяются параметры $\alpha, \beta_i, \gamma_i$. 

В результате, используя все полученные значения параметров, рассчитывается прогноз временного ряда из уравнения \eqref{arima}.Для этого нужно только выразить $y_{T+h}$, подставив $T+h$ вместо $t$, а также заменив все будущие значения их прогнозами, будущие ошибки --- нулями, а прошлые ошибки --- соответствующими остатками. Очевидно, что можно получать прогнозы только последовательно, например, для прогнозирования наблюдения $T+3$, где $T$ --- текущий момент времени, нам необходимо иметь прогноз для двух предыдущих периодов, чтобы подставить их на место ненаблюдаемых $y_{T+2}$ и $y_{T+1}$.

\subsection{Тета-метод} \label{sec:teta}
Представленный в работе \EN{Assimakopoulos, Nikolopoulos (2000)}, данный метод основан на прогнозировании с использованием декомпозиции. Однако, в отличии от классического подхода к декомпозиции, включающего в себя выделение сезонной компоненты, тренда и цикличности, в тета-методе производится выделение краткосрочной и долгосрочной компоненты. Прогнозом временного ряда в таких моделях является комбинация прогнозов каждой из компонент.

Упомянутые выше компоненты получаются в результате следующего преобразования исходного ряда $\lbrace y_t\rbrace$:\\
\begin{align*}
&\Delta^2 \widehat{y}_{t,\theta} = \theta \Delta^2 y_t , \quad \text{где}\\
&\Delta^2 y_t = y_{t} - 2 y_{t-1} + y_{t-2}
\end{align*}

Подставляя разные значения параметра $\theta$, можно получать различные, так называемые, тета-линии. Так, если установить этот параметр равным 0, мы получим прямую, которую авторы статьи рассматривают как долгосрочную компоненту исходного ряда. При $\theta > 1$ большую роль играют краткосрочные колебания данных, поэтому в оригинальной работе тета-линия с параметром 2 используется в качестве краткосрочной компоненты ряда. Получить прогноз первой, долгосрочной, компоненты, не доставляет труда, так как мы можем просто экстраполировать полученный линейный тренд. В случае со второй компонентой, авторы предлагают использовать для прогнозирования простое экспоненциальное сглаживание. Далее полученные прогнозы комбинируют, чтобы получить прогноз исходного ряда. В результате, прогноз на $h$ периодов вперед принимает следующий вид:
\begin{align*}
y_{T+h|T} = \frac12 (\widehat{y}_{T+h, 0} + \widehat{y}_{T+h, 2})
\end{align*}

В статье \EN{Hyndman and Billah} (2008) была предложена стохастическая модель пространства состояний для данного метода прогнозирования. 
\begin{align*}
y_t &= l_{t-1} + b + \varepsilon_t\\
l_t &= l_{t-1} + b + \alpha\varepsilon_t,
\end{align*}
где $\varepsilon_t \sim i.i.d N(0, \sigma^2)$, а начальное состояние задается как $y_1 = l_1$.

Пакет \texttt{forecast}, как и в случае с описанными выше одномерными моделями, предоставляет возможность автоматической оценки данной модели. Параметр $\alpha$, необходимый для оценки тета-линии с параметром $\Theta = 2$, определяется при помощи информационного критерия Акаике (см. уравнение \eqref{AIC}).

%%%%%%%%%%%%%%%%%%%%%%% PROPHET %%%%%%%%%%%%%%%%%%%%%%%
