\section{Обзор Литературы}
Ранние подходы к изучению временных рядов базировались на предположении об их детерминированности. То есть исследователи считали, что информация о временном ряде на некотором временном горизонте позволяет точно судить о его поведении вне этого интервала. Переход к стохастическому подходу произошел только в \textsc{XX} веке. Одной из поворотных статей в данной области стала работа Yule (1927), где временной ряд рассматривался как реализация некоторого случайного процесса. Впоследствии, основываясь на этой простой идее, ученые разработали огромное количество методов моделирования временных рядов.

\subsection{Одномерные модели временных рядов}
Наиболее широко используемыми подходами к прогнозированию одномерных временных рядов являются экспоненциальное сглаживание и модели ARIMA (autoregressive integrated moving average). Рассмотрим подробнее основную литературу, посвященную данным методам.

%% ETS
Модели экспоненциального сглаживания берут начало в литературе 1950-х годов. В статьях Holt (1957), Brown (1959) и Winters (1960) впервые использовался прогноз, представляющий собой среднее взвешанное всех доступных наблюдений с экспоненциально убывающими весами. В статье Muth (1960) впервые обсуждаются статистические свойства временных рядов, которые описываются моделью простого экспоненциального сглаживания. Автор показал, что такая модель является оптимальной для прогнозирования случайного блуждания с шумом.
%\begin{align*}
%y_t = \overline{y}_t +\eta_t \\
%\overline{y}_t = \overline{y}_{t-1} + \epsilon_t
%\end{align*}

В дальнейшем, большое количество работ было посвящено возможным расширениям простого экспоненциального сглаживания, самые популярные из которых --- метод Хольта с аддитивным трендом, метод Хольта-\\Уинтерса с аддитивным трендом и сезонностью (аддитивной или мультипликативной). В статье Hyndman, Koehler, Snyder, and Grose (2002) подробно рассматриваются все возможные спецификации моделей экспоненциального сглаживания, их формулировки в виде моделей пространства состояний с мультипликативными и аддитивными ошибками, а также сравнивается качество прогнозов, полученных в результате применения разных спецификаций к различным временным рядам.

%% ARIMA
Еще один крайне популярный способ прогнозирования временных рядов, ARIMA, представляет собой комбинацию авторегрессии (AR) и модели скользящего среднего (MA), допускающую взятие разности исходного ряда.  Внушительное число работ было посвящено выбору спецификаций модели, оценке коэффициентов и прогнозированию с использованием моделей из семейства ARIMA.  В результате, все основные, доступные на тот момент, научные достижения в этой области были объединены в работе Box \& Jenkins (1970).  В своей книге авторы подробно описывают разработанный ими подход к анализу временных рядов, известный до сих пор как методология Бокса-Дженкинса, который состоит из выбора подходящей модели, оценки параметров и проверки соблюдения всех необходимых предпосылок.

%% Комбинация прогнозов
По мере появления все большего числа моделей для прогнозирования временных рядов, исследователи стали задумываться об эффективности комбинации прогнозов различных моделей. Одной из первых работ на эту тему считается статья 	Bates and Granger (1969), где авторы представили несколько способов комбинации точечных прогнозов и показали, что такая техника способна улучшить качество прогноза по сравнению с результатами каждого метода по отдельности. На данный момент существует огромное количество способов комбинации прогнозов. Например, в статье Granger and Ramanathan (1984) было предложено оценивать веса прогнозов, используя метод наименьших квадратов на прошлых данных, а Deutsch, Granger, and Terasvirta (1994) предлагали использовать веса, изменяющиеся во времени.

\subsection{Векторная авторегрессия}
Для построения многомерных моделей одним из наиболее популярных инструментов является векторная авторегрессия (VAR). Представленная впервые в работе Sims (1980), данная модель отличается относительной простотой и не требует введения никаких ограничений на совместную динамику временных рядов. У VAR моделей существует большое количество спецификаций, благодаря чему их можно использовать для прогнозирования различных видов данных. Так, в статье Funke (1990) сравниваются результаты прогноза одномерной модели ARIMA и пяти разных спецификаций VAR модели (в частности структурной и байесовской VAR с различными априорными распределениями). Основной проблемой VAR моделей часто называют излишнюю параметризацию, которая может возникнуть при увеличении числа размерности модели или с ростом числа включаемых лагов и привести к невозможности получения оценок коэффициентов.

%% Про Байесовский вар
Для решения описанной выше проблемы \EN{Litterman} (1986) предложил ввести ограничение в виде априорного распределения параметров. Модели, в которых используется байесовский подход к оценке коэффициентов получили название BVAR (Bayesian Vector Autoregression). Главный недостаток байесовских моделей --- необходимость использования сопряженных распределений, перестал быть актуальным в современной науке благодаря высокой скорости, с которой современные компьютеры проводят симуляции и, как следствие, возможности применения методов MCMC\\ (Markov Chain Monte Carlo). Если же говорить о плюсах данного подхода, то он, в первую очередь, позволяет включить в модель большое число переменных и лагов, не опасаясь излишней параметризации. Более того, выбирая из широкого круга априорных распределений, исследователи имеют возможность снизить неопределенность относительно значений параметров, опираясь на опыт коллег или общепринятые представления о динамике тех или иных переменных.

Существует огромное количество работ, в которых BVAR используют для прогнозирования всевозможных временных рядов. Так, можно привести примеры прогнозирования макроэкономических данных \EN{(Artis and Zhang (1990);  Beauchemin, Kenneth and Zaman (2011); Berg,Oliver and Henzel (2013))}, рыночных долей компаний (Ramos (2003)), финансовых рядов \\(Carriero, Kapetanios and Marcellino (2012))

\subsection{Модели с переменными параметрами}
Модели, в которых параметры изменяются со временем, достаточно широко используются в макроэкономике и финансах. Такая тенденция вполне объяснима, так как закономерностям в данных отраслях свойственно изменяться по мере развития технологического прогресса, изменения конъюнктуры и перемен в обществе.
Одной из первых работ, в которой байсовская векторная авторегрессия применялась для прогнозирования временных рядов была \EN{Doan, Litterman and Sims} (1984). Авторы применили фильтр Калмана для обновления вектора параметров, представляющих собой процесс AR(1) и оценили значения гиперпараметров, при которых их модель давала наилучший прогноз среди других VAR моделей. 

По мере роста популярности байесовских методов стал расти и интерес к векторным авторегрессиям с переменными коэффициентами (TVP-\\VAR). Наиболее знаменитые работы в этой области --- \EN{Cogley and Sargent (2002, 2005), Primiceri (2005), Koop and Korobilis (2009)}. В первых трех статьях прогнозированию не уделялось особого значения, однако, в них были предложены основные принципы применения байесовского подхода к TVP-VAR моделям.

В статье \EN{Primiceri} (2005) была подробно рассмотрена байесовская векторная авторегрессия с переменными параметрами (далее --- TVP-BVAR).  У данной модели существует две основные особенности: во-первых, предполагается, что все коэффициенты модели могут изменяться со временем, а во-вторых, ковариационная матрица остатков также зависит от времени. Такие предположения позволяют строить модель для длительных временных промежутков, подстраиваясь под изменения конъюнктуры и меняющиеся зависимости между рассматриваемыми переменными. Как уже было сказано выше, автор статьи описывает применение байесовского подхода к подобной модели, а также предлагает алгоритм, позволяющий получить выборку из апостериорного распределения параметров. Однако, через несколько лет после публикации данной статьи в алгоритме была обнаружена ошибка, которую авторы исправили относительно недавно, опубликовав статью \EN{Primiceri, Del Negro} (2015).  Вскоре после этого была опубликована статья Kruger (2015), где был представлен пакет \texttt{bvarsv} для языка программирования \texttt{R}, позволяющий оценивать модель TVP-BVAR с использованием актуального алгоритма.
