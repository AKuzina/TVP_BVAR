\section*{Введение}
\label{sec:intro}
\addcontentsline{toc}{section}{\nameref{sec:intro}}

Прогнозирование временных рядов заключается в предсказании их будущих значений по настоящим и прошлым значениям при помощи каких-\\либо научных методов. Прогнозирование применяется в большом количестве ситуаций, например, для финансового и торгового планирования, при принятии управленческих решений на предприятиях или при выборе оптимальной макроэкономической политики Центральными банками. В некоторых случаях требуется прогнозирование на несколько лет вперед, например, при выборе крупного проекта для инвестирования, а иногда --- только на несколько минут, как при высокочастотной торговле на фондовом рынке. 

Выделяют два основных типа методов прогнозирования --- это качественные и количественные модели. К первым относится, например, использование экспертных оценок или проведение интервью. Однако, в текущей работе исследуются исключительно второй тип --- количественные методы, основанные на предположении, что закономерности, наблюдаемые в данных, могут быть экстраполированы в будущее.

Целью данной работы является сравнение качества прогнозов, которые получаются при использовании байесовской векторной авторегрессии с переменными параметрами, с прогнозами более распространенных одномерных методов, таких как ARIMA, экспоненциальное сглаживание, тета-метод и методов гибридных прогнозов для горизонтов различной длины и временных рядов из разных источников.  Для достижения указанной цели были поставлены следующие задачи:
\begin{itemize}
\item Сбор и первичная обработка финансовых и макроэкономических данных для последующего анализа
  \begin{itemize}
  \item Приведение данных к единому формату
  \item Проверка ряда на стационарность
  \item Разбиение выборки на тестовую и тренировочную
\end{itemize}
\item Оценка моделей на тренировочной выборке и построение прогнозов
  \begin{itemize}
  \item Выбор оптимальной спецификации модели
  \item Оценка всех неизвестных параметров 
  \item Построение прогнозов будущих значений рядов
  \item Расчет ошибок прогноза на тестовой выборке
\end{itemize}
\item Сравнение полученных результатов для разных моделей
\end{itemize}

Как уже было сказано выше, задача прогнозирования временных рядов имеет высокое значение для различных сфер экономики, а в силу большого числа доступных статистических методов не менее важной задачей становится выбор модели, которая лучше всего справится с прогнозированием. В силу ограниченности ресурсов, на практике выбор из огромного множества доступных моделей осуществляется на основе результатов, которые данная модель продемонстрировала в различных исследованиях. Именно поэтому необходимы работы, в которых новые методы сравниваются с более привычными или простыми моделями на различных данных.

Байесовская векторная авторегрессия достаточно часто применяется при прогнозировании макроэкономических показателей, так как позволяет включать в модель большое количество переменных и их лагов, а также учитывать априорные представления о распределении параметров модели. Использование моделей с переменным параметрами дает еще более широкие возможности для анализа совместной динамики переменных, хотя и значительно усложняет процедуру оценки модели. В данной работе исследуется именно такая модель и используется относительно новый алгоритм семплирования из статьи Primiceri, Del Negro (2015), призванный устранить ошибку в алгоритме, который использовался в более ранних работах.

Первая часть работы посвящена обзору литературы, относящейся к выбранным методам прогнозирования. Во второй части подробно описаны используемые модели, их возможные спецификации и способы оценки параметров. Третья часть описывает данные, которые были выбраны для проведения анализа, методологию построения прогнозов и их сравнения между собой, а также полученные результаты.
