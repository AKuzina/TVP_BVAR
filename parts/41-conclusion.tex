\section*{Заключение}
\label{sec:concl}
\addcontentsline{toc}{section}{\nameref{sec:concl}}

В данной работе было проведено исследование прогнозов, построенных с помощью байесовской векторной авторегрессии с переменными параметрами и одномерных моделей, таких как ARIMA, экспоненциальное сглаживание, тета-метод. Целью работы было сравнить качество полученных прогнозов для различных типов данных и для горизонтов различной продолжительности. Для анализа было выбрано два набора данных: месячные макроэкономические показатели и дневные изменения цен акций крупнейших российских компаний.

После оценки моделей на тренировочной выборке были построены прогнозы и рассчитаны ошибки моделей по сравнению с моделью случайного блуждания. Мы увидели, что прогнозы TVP-BVAR оказались ближе всего к истинным значениям ровно в половине случаев для макроэкономических данных и более чем в трети случаев для финансового набора данных. Тем не менее, было замечено, что для финансовых данных разница в относительных ошибках незначительна, в результате чего нельзя однозначно говорить о преимуществе данной модели.

Серьезным ограничением при проведении данного исследования служили возможности используемого статистического пакета. Одним из возможных расширений данной работы могла бы служить оптимизация кода, которая позволила бы оценивать модель с большим количеством переменных и лагов. Предположительно, это позволило бы получить более точные прогнозы модели TVP-BVAR, так как на макроэкономических данных, где для каждой переменной было построено несколько многомерных моделей, «большие» модели оказались более эффективными.
