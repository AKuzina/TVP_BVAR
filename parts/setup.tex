\usepackage{etex}                % расширение классического tex
\usepackage{cmap}                % для поиска русских слов в pdf
\usepackage{verbatim}            % для многострочных комментариев
\usepackage{makeidx}             % для создания предметных указателей

%%% Кодировки и шрифты %%%
\usepackage{polyglossia}

\setmainlanguage[babelshorthands=true]{russian}
\setotherlanguages{english}

\setmainfont{Cambria}             % or Helvetica, Arial, Cambria
\setmonofont{Courier New}
\newfontfamily{\cyrillicfonttt}{Courier New}
% why do we need \newfontfamily:
% http://tex.stackexchange.com/questions/91507/

\defaultfontfeatures{Ligatures=TeX,Mapping=tex-text}


\IfFileExists{pscyr.sty}{\usepackage{pscyr}}{}  % Красивые русские шрифты


%% Библиография

%% Внимание! При использовании bibtex8 необходимо удалить все
%% цитирования из  ../common/characteristic.tex
\newcounter{bibliosel}
\setcounter{bibliosel}{1}           % 0 --- встроенная реализация с загрузкой файла через движок bibtex8; 1 --- реализация пакетом biblatex через движок biber

%%% Поля и разметка страницы %%%

\usepackage{pdflscape}             % Для включения альбомных страниц
\usepackage{geometry}              % Для последующего задания полей
\geometry{a4paper,top=2cm,bottom=2cm,left=2.5cm,right=1cm,nomarginpar}

\usepackage{indentfirst}           % Красная строка
\setlength{\parindent}{2em}
\setlength{\topsep}{0pt}

\usepackage{setspace}              % Межстрочный интервал
\setstretch{1.5}                   

\flushbottom                       % Эта команда заставляет LaTeX чуть растягивать строки, чтобы получить идеально прямоугольную страницу
\righthyphenmin=2                  % Разрешение переноса двух и более символов
\pagestyle{plain}                  % Нумерация страниц снизу по центру.
\widowpenalty=50000               % одна строка абзаца на этой странице, остальное --- на следующей
\clubpenalty=50000                % одинокая строка в начале страницы

%%% Математические пакеты %%%
\usepackage{amsthm,amsfonts,amsmath}
\usepackage{amssymb,amscd}        % Математические дополнения от AMS
\usepackage{mathtools}            % Добавляет окружение multlined





%%% Окружение для нормального центрирования %%%
\newenvironment{center*}{%
  \setlength\topsep{0pt}
  \setlength\parskip{0pt}
  \begin{center}
}{
  \end{center}
}


%%% Таблицы %%%
\usepackage{longtable}             % Длинные таблицы
\usepackage{multirow,makecell}     % Улучшенное форматирование таблиц
\usepackage{array,multicol,bigstrut}
\usepackage{tabularx}      % таблички растягиавать
\usepackage{tabulary}
\usepackage{booktabs}


%%% Гиперссылки %%%
\usepackage[unicode,colorlinks=true,urlcolor=blue,hyperindex,breaklinks]{hyperref}

%%% Списки %%%
\usepackage{enumitem}

%%% Подписи %%%
\usepackage{caption}                % Для управления подписями
\usepackage{subcaption}             % Работа с подрисунками и подобным
\usepackage{totpages}               % Счётчик страниц


%%% Для добавления Стр. над номерами страниц в оглавлении
%%% http://tex.stackexchange.com/a/306950
\usepackage{afterpage}

%%% Переопределение именований, чтобы можно было и в преамбуле использовать %%%
% \renewcommand{\chaptername}{Глава}
% \renewcommand{\appendixname}{Приложение}
% \newcommand{\fullbibtitle}{Список литературы}